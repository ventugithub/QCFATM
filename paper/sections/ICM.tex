\subsection{Solving QUBO instances using ICM}

In this section we present the results for the optimization of the QUBO
formulation of the ATM problem using the Isoenergetic Cluster Method (a
rejection-free cluster algorithm for spin glasses that greatly improves
thermalization)~\cite{zhu2015}, which has been shown to be one of the fastest
classical heuristic to optimize QUBO problems~\cite{mandra2016}.\\

Figure~(\ref{fig:icm1}) shows the total delay time optimized by ICM either by 
varying the partition at fixed the delay step $\Delta t$ (left panel) or by
varying the delay step $\Delta t$ at fixed partition (right panel). As one can
see, the total delay decreases by decreasing $\Delta t$ and it eventually
reaches an optimal plateau. Results are for maximum delay of 60 minutes. This is
consistent with the idea that smaller $\Delta t$ allows a finer optimization of
the delays of the flights.\\

\begin{figure*}
  \includegraphics[width=\columnwidth]{pics/qubo_icm/qubo_icm_3.pdf}
  \includegraphics[width=\columnwidth]{pics/qubo_icm/qubo_icm_4.pdf}
  \caption{(Left) Optimal total delay found by using the
  Isoenergetic Cluster Method (ICM) at fixed time step $\Delta t$, by varying
  the connected component. Results are for maximum delay time of $60$ minutes. (Right)
  Optimal delay found by using ICM at fixed connected component, by varying the time step
  $\Delta t$.}
\label{fig:icm1}
\end{figure*}

In Figure~(\ref{fig:icm2}) we show the optimal delay time found by ICM as a
function of the number of the flights in the connected components. Results are
for a maximum delay of 60 minutes. Unfortunately, ICM was unable to optimize
connected components with more than $12$ flights. This can be explained by
recalling that ICM works the best for almost-planar problem while the
its performance quickly decreases for fully-connected problems. Indeed, as shown
in Section~\ref{sec:instances}, the underlying graph of connected components
look more like a fully-connected graph rather than a tree graph by increasing
the number of flights inside the connected component.

\begin{figure}
  \includegraphics[width=\columnwidth]{pics/qubo_icm/qubo_icm_2.pdf}
  \caption{\label{fig:icm2}. Optimal total delay found by using the Isoenergetic
  Cluster Method (ICM) at fixed time step $\Delta t$ as a function of numbers of
  flight within each connected component. ICM was unable to find solutions for connected
  component with more than $12$ flights.}
\end{figure}
