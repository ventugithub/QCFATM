\subsubsection{Maneuvers}
A more realistic model of the problem can be created by including maneuvers.
As mentioned above the maneuvers enter our formulation as additional delays $d_{ik}$ at the conflict time.
In the course of mapping to a QUBO formulation, we need to make sure to retain the combinatorial nature of the problem.
We do this by restricting the vast realm of maneuvers to two distinct choices:
Only one of the two involved flights is delayed while leaving the other flight untouched
\begin{equation} \label{eqn:maneuver_model_maneuver_decision}
    \text{if } d_{ik} \neq 0 \Rightarrow d_{jk} = 0  \qquad \forall (i, j) \in I_k \; \forall k \; .
\end{equation}
Moreover, we set the resulting maneuver delays to a constant value $d_M$ large enough to capture all kinds of real maneuvers.
A natural choice for this is the temporal conflict threshold $d_M = \Delta_t$.

With~\eqref{eqn:time_dependent_delay} we can introduce the delay a flight $i$ at the conflict $k$ as
\begin{equation} \label{eqn:maneuver_model_delay_at_conflict}
    D_{ik} = d_i + \sum_{k'<k} d_{ik} \; ,
\end{equation}
where we have defined a temporal ordering of the conflicts for each flight $i$ by
\begin{align*}
                 &k < p \text{ if } t < t' \\
    \text{ for } &t = \min_s x_{i, s} \in C_k \; , \\
                 &t' = \min_s x_{i, s} \in C_p
\end{align*}

The departure delay variables are represented by binary variables as it was done in Section~\ref{sec:departure_delay_model}.
The maneuver delays are given by
\begin{equation*}
    d_{ik} = d_M a_{ik} \qquad a_{ik} \in \{0, 1\}
\end{equation*}
Since the total delay is given by $\sum_{ik} D_{ik}$, we can write the corresponding QUBO contribution as
\begin{equation*}
    \tilde Q_\text{delay} = \sum_{i\alpha} \alpha d_{i\alpha}  + \sum_{ik} d_M a_{ik}\; ,
\end{equation*}

For the conflict avoidance, we need to introduce another variable representing the delay at a given conflict
\begin{equation*}
    D_{ik} = \sum_\delta \delta \Delta_{ik\delta} \qquad \Delta_{ik\delta} \in \{0, 1\} \; .
\end{equation*}
By restricting ourselves to $\Delta_d = \Delta_t$ the values of $\delta$ in the above equation are given as
\begin{equation*}
    \delta \in \{0, \Delta_t, 2\Delta_t, \dots,  (N_d + M_{ik}) \Delta_t\} \; .
\end{equation*}
Here, $M_{ik}$ is the number of conflicts the flight $i$ is involved in before $k$.
In order to fulfill~\eqref{eqn:maneuver_model_delay_at_conflict} we add the following contribution to the QUBO
\begin{equation*}
  \tilde Q_\Delta = \lambda_\Delta \sum_{ik}  {\left( \sum_{\alpha} \alpha d_{i\alpha}  + \sum_{k'<k} d_M a_{ik'} - \sum_\delta \delta \Delta_{ik\delta}\right)}^2 \biggl. \biggr|_{i, j \in I_k}
\end{equation*}
For unique representation of the variables we add 
\begin{align*}
  \tilde Q_\text{unique} = \lambda_\text{unique} & \left\{  \sum_i {\left( \sum_\alpha d_{i\alpha} - 1 \right)}^2 \right. \\
  & \left. + \sum_{ik} {\left( \sum_\delta \Delta_{ik\delta} - 1 \right)}^2 \right\} \; .
\end{align*}
Conflicts are avoided if $D_{ik} - D_{jk} \notin D_k$, $(i, j) \in I_k$. 
The corresponding QUBO contribution reads
\begin{equation*}
    \tilde Q_\text{conflict} = \lambda_\text{conflict} \sum_k \sum_{(\delta, \delta') \in B_k} \Delta_{ik\delta} \Delta_{jk\delta'} \biggl. \biggr|_{i, j \in I_k}
\end{equation*}
where $B_k$ is the set of all $(\delta, \delta')$ which correspond to a conflict
\begin{equation*}
    B_k = \{(\delta, \delta') \; | \; \delta - \delta' \in D_k\}
\end{equation*}
The penalty weights $\lambda_\Delta$, $\lambda_\text{unique}$ and $\lambda_\text{conflict}$ must be chosen large enough to ensure vanishing contributions from the corresponding QUBO terms for the solution.

Finally, the maneuver decision described by~\eqref{eqn:maneuver_model_maneuver_decision} is incorporated by a antiferromagnetic coupling between the two maneuver delay variables
\begin{equation*}
  \tilde Q_\text{maneuver} = J \sum_k  {\left( s_{ik} s_{jk} + 1\right)}_{i, j \in I_k} \; .
\end{equation*}
with 
\begin{equation*}
    s_{ik} = 2 a_{ik} - 1 \in \{-1, 1\}
\end{equation*}
and $J>0$ has to be chosen large enough. 
A solution is considered to be valid only if $\tilde Q_\text{maneuver} = 0$.
Hence, the total QUBO for the maneuver model reads
\begin{equation*}
    Q_\text{MM} = \tilde Q_\text{delay} + \tilde Q_\Delta  + \tilde Q_\text{unique} + \tilde Q_\text{conflict} + \tilde Q_\text{maneuver} 
\end{equation*}
