\section{Conclusions}
In this paper, we propose a novel QUBO mapping for a simplified version of one
the NASA critical missions: the Air Traffic Management (ATM), i.e. the problem
to find minimal modifications of wind-optimal trajectories to avoid conflicts
between flights. In our study, we considered the actual wind-optimal
trajectories for transatlantic flights (NAT) on July 29, 2012. Given the large
number of flights, it is not a viable solution to directly map the optimal-wind
trajectories in a QUBO model. To avoid this bottleneck, our modified version of
the ATM problem is based on the main assumption that flight maneuvers to avoid
conflicts only locally modify the optimal-wind trajectories, with the net effect
to introduce ``delays'' to the flights. Therefore, optimal-wind trajectories can
be ``hard encoded'' in our QUBO formulation of the ATM model, leaving the
flights delays as the only variables to optimize. Nevertheless, as explained in
Appendix 2, our method is enough general to potentially include the effect of
maneuvers as well. 

As part of our study, we also introduce a novel ``pre-processing'' algorithm to
eliminate non-potential conflicts that, given a maximum delay, can never occur.
This novel approach is not only important to greatly reduce the number of
potential conflicts (as shown in Section XX), but it also gives an important
indication of the underlying topology the conflict graph. Indeed, we have
discovered that most of the flights have very few conflicts while there are
few flights that have conflicts in a non trivial way. The latter set of flights
represent the hardest part of the ATM problem to optimize. We want to stress
that the proposed pre-processing algorithm is general and can be successfully
applied for the original ATM problem as well, aiding the already existing
software to improve both the speed and quality to find optimal modification of
the wind-optimal trajectories. 

Finally, we have analyzed the performance of both classical and quantum
heuristics in solving the our QUBO model where only delays at the departure are
allowed. Results show that xxx [what can we say here guys?]

In conclusion, we present one of the first attempt to model the Air Traffic
Management problem onto a QUBO problem. 


{
\color{red} 
This work represents the foundation for future work, including
\begin{itemize}
    \item Embed and solve QUBO instances for models including maneuvers on a quantum annealer.
    \item Improve performance of quantum annealing by alternative embedding strategies and newer D-Wave devices.
    \item Solve larger subsets of the problem with classical solvers and compare to other classical approaches.
\end{itemize}
}
