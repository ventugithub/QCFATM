\section{Introduction}
There is an overall increase in air traffic over the last decades and this trend is believed to continue.
As a result, the air traffic control system is increasingly under pressure.
The current system works with pre-planned routes for the flights and more or less manual conflict avoidance with the help of air traffic controllers.
With the limited airspace available, novel approaches are necessary to meet the demand of even more air traffic in the coming decades.
A promising approach to solve this problem is using wind-optimal trajectories beyond the predefined air traffic network \cite{ng_optimizing_2014}.
However, the conflicts in the wind-optimal trajectories need to be avoided, or ``deconflicted``  \cite{rodionova16}.

Quantum annealing is a promising computational method which became increasingly important in recent years.
This development is driven also by first commercially available quantum annealing device by the company D-Wave Systems.
In addition to studying the fundamental properties of quantum annealing, it is imperative to find possible real world application for this technology.
Hard operational planning problems are a promising candidate for the latter \cite{Rieffel2015, Venturelli2015}.

In this work, we investigate the feasibility of applying quantum annealing to the deconfliction of wind-optimal trajectories. 
In the course of this, we use wind-optimal trajectory data to extract realistic problem instances.
To be amenable to a D-Wave Quantum annealer, the problem has to be formulated as a quadratic unconstrained binary optimization (QUBO) problem.
We demonstrate the mapping of deconfliction problem to a QUBO formulation.
Since restrictions to the configurations space are necessary for the reformulation of the problem as a QUBO, we investigate the influence of this restriction on the solution quality.
Moreover, we study the embeddability and solution quality using both classical solvers and quantum annealing runs.

