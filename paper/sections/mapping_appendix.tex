\section{General QUBO mapping}
In this section we describe a mapping to QUBO of a more general version of the deconflicting problem than that covered in the main text.

\subsection{Alternative encodings}
In the mappings describe both in the main text and the appendix, we use a one-hot encoding to encode a variable.
This is best for the specific mappings we described, but in variants an alternative may be better.
Say we have a variable $x$ that we want to allow to have variables from finite set $W = \{w_1, w_2, \ldots, w_m\}$. 
The one-hot encoding has $m$ bits ${\left(x_{i}\right)}_{i=1}^m$ such that $x = \sum_{i=1}^m w_i x_i$ and $\sum_{i=1}^m x_i = 1$.
While we focus on the case in which $W = \{0, 1, \ldots, m -1\}$, our methods are not dependent on that being case, and in particular can address non-uniform sets of values, say if via clever preprocessing it can be determined that such a set would be sufficient.
An alternative encoding would remove the requirement that exactly one of the bits is one.
The variable $x$ would still be encoded as $x = \sum_{i=1}^m w_i x_i$, but without the one-hot constraint can take on values in $\left\{\sum_i b_i w_i \middle| b_i \in \{0, 1\}\right\}$.
In particular, this encompasses the unary encoding in which $w_i = 1$ for all $i$ and thus $x \in [0, m]$, as well as the binary encoding $w_i = 2^{i-1}$ for which $x \in [0, 2^m - 1]$.
The latter has the advantage of requiring much fewer qubits, but at the cost of similarly increased precision.
The former requires the same number of qubits as the one-hot encoding we use, and even has the benefit of minimal precision, but does not allow for quadratic constraints that penalize certain pairs of values of variables, e.g. $d_i - d_j \neq B_k$, without the use of ancillary bits.
In models in which the bits $x_{i}$ only appear in the sum $\sum_i x_i$, it is actually preferable to use the unary encoding to improve the precision requirements.
We stick to the one-hot encoding for simplicity, but in practice the unary encoding should be used when possible.

To make the expressions more concise, we define the generalized encoding penalty function
\begin{equation}
\function{encoding}
\left(
{\left\{X_i\right\}}_i
\right)
=
\weight{encoding}
\sum_i 
{\left(
  \sum_{x \in X_i} x - 1
\right)}^2
\end{equation}
that enforces the constraint that exactly one bit $x$ is one for each set of bits $X_i$.

\subsection{Global trajectory modifications}
Consider the case in which each trajectory can be modified by a departure delay and some parameterized spatial transformation, i.e.\ for each flight $i$ there is a variable $d_i$ and some parameter $\boldsymbol \theta_i$.
For example, Rodionova et al.~\cite{rodionova16} consider a single angle $\theta_i$ that determines a sinusoidal transformation of the trajectory.
For the QUBO mapping, we require that these variables be allowed to take on values from some finite set, so that are QUBO variables are $\{d_{i, \alpha}\}$ and $\{\boldsymbol \theta_{i,\phi}\}$, where $d_{i, \alpha} = 1$ ($d_{i, \alpha}$) indicates that $d_i = \alpha$ ($d_i \neq \alpha$) and similarly for $\boldsymbol \theta_{i, \phi}$.
For every pair of flights $i < j$, we can efficiently (in time and space polynomial in the size of the input) compute whether the corresponding trajectories conflict when modified according to $d_i$, $d_j$, $\boldsymbol \theta_i$ and $\boldsymbol \theta_j$.
Let $B_{i, j}$ be the set of values of $(d_i, \boldsymbol \theta_i, d_j, \boldsymbol \theta_j)$ such that the the modified trajectories conflict.
Lastly, let $d_{(i, \alpha), (j, \beta)}$ indicate that $d_i = \alpha$ and $ d_j = \beta$, and similarly for $\boldsymbol \theta_{(i, \phi), (j, \psi)}$.
The overall cost function is
\begin{multline}
\function{global}
\left(
{\left(d_{i,\alpha}\right)}_{i,\alpha}
{\left(d_{(i,\alpha),(j,\beta)}\right)}_{i,j,\alpha,\beta}
{\left(\boldsymbol \theta_{(i,\phi),(j,\psi)}\right)}_{i,j,\phi,\psi}
\right)
=\\
\function{encoding} +
\function{consistency} +
\function{delay} +
\function{conflict},
\end{multline}
where 
\begin{equation}
\function{encoding} \left(
{\left\{
{\left\{
d_{i, \alpha}
\right\}}_{\alpha}
\cup
{\left\{
\boldsymbol \theta_{i, \phi}
\right\}}_{\phi}
\right\}}_i
\right)
\end{equation}

%\begin{equation}
%  \function{encoding}
%  =
%  \weight{encoding}
%  \left[
%  \sum_{i} {\left(\sum_{\alpha} d_{i, \alpha} -1\right)}^2
%  +
%  \sum_{i} {\left(\sum_{\phi} \boldsymbol \theta_{i, \phi} -1\right)}^2
%\right]
%\end{equation}
ensures that the values of $d_i$ and $\boldsymbol \theta_i$ are uniquely encoded;
\begin{multline}
\function{consistency}
=\\
\begin{split}
\weight{consistency}
\bigg[
&\sum_{i < j,\alpha, \beta} 
s\left(d_{i,\alpha}, d_{j, \beta}, d_{(i, \alpha), (j, \beta)}\right)
\\
&+
\sum_{i < j,\phi, \psi} 
s\left(\boldsymbol \theta_{i,\phi}, \boldsymbol \theta_{j, \psi}, \boldsymbol \theta_{(i, \phi), (j, \psi)}\right)
\bigg]
\end{split}
\end{multline}
ensures consistency between the values of $d_{i,\alpha}$, $d_{j, \beta}$, and $d_{(i, \alpha), (j, \beta)}$;
\begin{equation}
s(x, y, z) = 3z + xy - 2xz - 2 yz
\end{equation}
is a non-negative penalty function that is zero if and only if $z = xy$;
\begin{equation}
\function{delay}
= \sum_{i, \alpha} \alpha d_{i, \alpha}
\end{equation}
is the cost function to be minimized; and
\begin{equation}
  \function{conflict}
=
\weight{conflict}
\sum_{i < j} 
\sum_{(\alpha, \phi, \beta, \psi) \in B_{i, j}} 
d_{(i, \alpha), (j, \beta)} \boldsymbol \theta_{(i, \phi), (j, \psi)}
\end{equation}
penalize conflicts.

\subsection{Local trajectory modifications}

Alternatively, we can consider modifications to the trajectory only near conflicts.
We describe a few special models and their mapping to QUBO,
though many more such ways of doing so, and we leave a full accounting for future work.

\subsubsection{Exclusive avoidance}
Suppose for every conflict $k$ and associated pair of flights $i < j$, there is a way for either flight to go around the trajectory of the other, introducing some delay $d_{i,k}$ to flight $i$ or $d_{j, k}$ to flight $j$ depending on which trajectory is changed.
Let $a_k = a_{i, k} = 1$ ($a_{i, k} = 0$) indicate that flight $i$'s trajectory is changed, and for convenience let $a_{j, k} = a_{i, k} - 1$, though only one (qu)bit will be used per conflict. % chktex 36
Adding in the departure delay, we have the total cost function
\begin{equation}
\function{exclusive}
\left(
{\left(
  d_{i, \alpha}
\right)}_{i, \alpha},
{\left(a_k\right)}_k
\right)
=
\function{delay} +
\function{encoding},
\end{equation}
where 
\begin{equation}
\label{eq:delay-exclusive}
\function{delay} = 
\sum_{i}
\left[
\sum_{\alpha} \alpha d_{i, \alpha}
+
\sum_{k \in K_i} d_{i, k} a_{i, k}
\right]
\end{equation}
and
$\function{encoding}$ is as in~\eqref{eq:dep-delay-encoding-penalty}.
This assumes that the trajectory modifications don't introduce potential conflicts with other flights; this assumption can be partially relaxed by adding penalty terms of the form $a_{i,k} a_{j,k'}$ or $d_{i,\alpha} a_{j, k}$ as appropriate.

\subsubsection{Flexible avoidance}
Exclusive \emph{requires} that one or the other flight is delayed at each conflict.
We can relax this by accounting for the fact that if the flights arriving at a potential conflict are already relatively delayed, the conflict could be passively avoided.
Let $D_{k, \gamma} = 1$ ($D_{k, \gamma} = 0$) indicate that $D_k = \gamma$ ($D_k \neq \gamma$), where $D_k$ is the difference in the accumulated delays at conflict $k$ as defined in~\eqref{eq:accum-delay-diff}.

The total cost function is
\begin{multline}
\function{flexible}
\left(
{\left(d_{i, \alpha}\right)}_{i, \alpha},
{\left(a_{i, k}\right)}_{k, i \in I_k},
{\left(D_{k, \gamma}\right)}_{k, \gamma}
\right)
=\\
\function{encoding} +
\function{delay} +
\function{consistency} + 
\function{conflict},
\end{multline}
where the first term is
\begin{equation}
\function{encoding}
\left(
{\left\{
{\left\{
d_{i, \alpha}
\right\}}_{\alpha}
\right\}}_i
\cup
{\left\{
{\left\{
D_{k, \gamma}
\right\}}_{\gamma}
\right\}}_{k}
\right);
\end{equation}
%\begin{equation}
%\function{encoding}
%=
%\weight{encoding}
%\left[
%  \sum_i {\left(\sum_{\alpha} d_{i, \alpha} - 1\right)}^2
%  +
%  \sum_{k}
%  {\left(\sum_{\gamma} D_{k, \gamma} - 1\right)}^2
%\right]
%\end{equation}

the consistency term is
\begin{equation}
\function{consistency}
=
\weight{consistency}
\sum_k
{\left( 
D_{i, k} - D_{j, k}
-
\sum_{\gamma} \gamma D_{k, \gamma}
\right)}^2
\end{equation}
using the notational variables
\begin{equation}
D_{i, k} = \sum_{\alpha} \alpha d_{i, \alpha} +
\sum_{k' \in K_{i, k}}
d_{i, k'} a_{i, k'};
\end{equation}
$\function{delay}$ is as in~\eqref{eq:delay-exclusive} but where $a_{i,k}$ and $a_{j, k}$ are separate bits;
and
\begin{equation}
\function{conflict}
=
\weight{conflict}
\sum_k \sum_{\gamma \in B_k} 
\left[
D_{k, \gamma}
\left(1 - a_{i, k} - a_{j, k}\right)
+ 2 a_{i, k} a_{j, k}
\right]
\end{equation}

If we want to allow both flights to be delayed at conflict $a_{i,k} = a_{j, k} = 1$, we must introduce an ancillary bit $a_k$ that indicates whether at least one flight is delayed at conflict $k$, adding
\begin{equation}
  \weight{consistency}
  \sum_{k} 
  \left[
    \left(a_{i, k} + a_{j, k} \right) \left(1 - 2 a_k\right)
    + a_{i, k} a_{j, k}
  \right]
\end{equation}
to $\function{consistency}$, and
replacing $\function{conflict}$ with
\begin{equation}
\sum_k \sum_{\gamma \in B_k} D_{k, \gamma} (1 - a_k).
\end{equation}

\subsubsection{Interstitial delays}

In the interstitial-delay model, the local modifications are not made \emph{at} conflicts but \emph{between} them, and conflicts are only avoided via accumulated delays.
That is, the delay $d_{i, k}$ introduced to flight $i$ before reaching conflict $k$ but after leaving the previous conflict $\max_{k' \in K_{i, k}}k'$.
Unlike in the flexible avoidance model, $d_{i,k}$ is now a variable rather than a parameter, and we encode it using bits $d_{i, k, \delta}$.
\begin{multline}
\function{interstitial}
\left(
{\left(
d_{i, \alpha}
\right)}_{i, \alpha},
{\left(
D_{i, k, \gamma}
\right)}_{i, k \in K_i, \gamma}
\right)
=\\
\function{encoding}
+
\function{consistency}
+
\function{conflict}
+
\function{delay},
\end{multline}
where
\begin{equation}
\function{encoding}
\left(
{\left\{
{\left\{
d_{i, \alpha}
\right\}}_{\alpha}
\right\}}_i
\cup
\bigcup_i
{\left\{
{\left\{
D_{i, k, \gamma}
\right\}}_{\gamma}
\right\}}_{k \in K_i}
\right),
\end{equation}

%\begin{equation}
%\function{encoding}
%=
%\weight{encoding}
%\left[
%\sum_{i=1}^{\Nf}
%{\left(
%\sum_{\alpha} d_{i, \alpha}
%- 1
%\right)}^2
%+
%\sum_{i=1}^{\Nf}
%\sum_{k \in K_i}
%{\left(
%\sum_{\delta} d_{i, k, \delta} 
%- 1
%\right)}^2
%+
%\sum_{k=1}^{\Nc}
%{\left(
%\sum_{\delta} D_{i, k, \gamma} - 1
%\right)}^2
%\right]
%\end{equation}

\begin{equation}
\function{consistency}
=
\sum_i
\sum_{\left.k, k' \in K_i \middle| k' = \max K_{i, k}\right.}
\sum_{(\gamma, \gamma') \in B_{i, k}}
D_{i, k, \gamma} D_{i, k', \gamma'},
\end{equation}

%\begin{widetext}
%\begin{equation}
%\function{consistency}
%=
%\weight{consistency}
%\sum_{i=1}^{\Nf}
%\sum_{k \in K_i}
%{\left(
%\sum_{\alpha} \alpha d_{i, \alpha}
%+
%\sum_{k' \in K_{i, k}}
%\delta d_{i, k, \delta}
%-
%\sum_{\gamma} \gamma D_{i, k, \gamma} 
%\right)}^2
%\end{equation}
%\end{widetext}

\begin{equation}
\function{conflict}
=
\weight{conflict}
\sum_{k=1}^{\Nc}
\sum_{(\gamma, \gamma') \in B_k} D_{i, k, \gamma} D_{j, k, \gamma'},
\end{equation}
and
\begin{equation}
\function{delay}
\sum_i \sum_{\gamma} D_{i, \max K_i, \gamma}.
\end{equation}
