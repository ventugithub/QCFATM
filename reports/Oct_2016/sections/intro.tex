\section*{Introduction}\label{sec:intro}
Building on our promising prior results in the planning domain~\cite{rieffel:15,venturelli:15},
we have begun to explore the potential of quantum annealing (QA) for solving challenging computational problems related to air traffic management (ATM)\cite{rodionova:16, rodionova:thesis15}.
This work is being performed by members of the QuAIL team (Tobias Stollenwerk, Bryan O'Gorman, Salvatore Mandr\`a, Davide Venturelli and Eleanor G. Rieffel) with expertise in all aspects of applying quantum annealing to real-world problems, in close collaboration with domain experts in ATM (Olga Rodionova, Hok K. Ng and Banavar Sridhar).
We have identified a specific problem within ATM to serve as a case study, and have made significant progress towards formulating it in a way that is amenable to quantum computing.\\

In order to optimize our formulation of the problem to realistic data, we focus on flight data in the North Atlantic oceanic airspace (NAT), for which we have wind-optimal trajectories for two consecutive days (July 28\textsuperscript{th}-29\textsuperscript{th} 2012) for which state-of-the-art solutions exist. 
The NAT dataset consists of wind-optimal trajectories in (3+1)-dimensions for 984 flights. 
These trajectories, subsets thereof, and toy-instances based thereon will serve as our benchmark set.


