\section*{Outlook and next steps}\label{sec:ass}

Already we have formulated the ATM problem in a way that is amenable to pseudo-Boolean optimization.
Our next step is to finalize the mappings to QUBO, and assess the quality of solutions from the resulting QUBO instances to those found by prior classical methods, based on ATM instances drawn from the NAT dataset.
That is, before we analyze the performance of quantum annealing in solving our formulation of the problem, we will analyze the formulation itself, by comparing solutions from classical QUBO solvers to those from extant methods.
We are interested not only in solutions that are at least as good in quality, but those that are of at least similar quality but sufficiently different from those produced by other methods; in computational models of real-world problems such as this, there is a wide-ranging interest in a \emph{diversity} of solutions from which a human can select based on unformalized criteria.

We will also construct small instances, either derived directly from subsets of the NAT dataset or based on the structures therein, and run them both on a physical state-of-the-art D-Wave quantum annealer as well as a simulation of quantum annealer, which allows for larger sizes in order to asses the scalability of our methods.
Finally, we will explore the resource requirements of various mappings for realistic data in order to direct future hardware designs.
These last steps encompass the motivating premise of this work: that by formulating computational problems of interest to NASA in a way that is amenable to quantum annealing, we can set the stage for both directing and utilizing the development larger-scale quantum annealers.

For a complete overview of completed tasks and future milestones, see \tablename~\ref{table:milestone}.
\figurename~\ref{fig:scheme} summarizes the steps involved in applying quantum annealing to the ATM problem.

\noindent
\textbf{Project outputs:}
\begin{itemize}
  \item A programming toolkit to analyze trajectories and potential conflicts, with wide-ranging utility.
  \item A variety of mappings of the ATM problem to QUBO.
  \item Implementation of the mappings, taking a set of wind-optimal trajectories and outputting a QUBO instance.
  \item Runs of toy instances of the ATM problem on the D-Wave quantum annealer housed at NASA Ames.
  \item A publication assessing the viability of the problem formulation developed.
\end{itemize}

\begin{table}[h!]\centering
  \begin{tabular}{|m{7cm}|m{7cm}|K{2.8cm}|}
    \rowcolor{gray!30}
    \hline
    \multicolumn{1}{|c|}{\textbf{Task/Milestone}} & \multicolumn{1}{c|}{\textbf{Performance Metric}} & \textbf{Expected completion from start of project}\\
    \hline
    \rowcolor{green!30}
    Implement code to identify potential conflicts for different thresholds. &
        Largest set of trajectories that the code can analyze. Overall performance of the code. & 1.5 month \\
    \hline
    \rowcolor{green!30}
    Implement a toolkit to visualize and analyze the structure of potential conflicts.
      & Usefulness in formulating a model of potential maneuvers. & 2 month \\
    \hline
    \rowcolor{green!30}
      Map the ATM problem without maneuvers to QUBO\. & 
    Quality and diversity of solutions resulting from QUBO\.
    Physical resource requirements (number of qubits, connectivity of logical graph, embeddability into hardware). & 3 months\\
    \hline
    \rowcolor{orange!50}
    Map the ATM problem with maneuvers to QUBO\. & 
    Quality and diversity of solutions resulting from QUBO\.
    Physical resource requirements (number of qubits, connectivity of logical graph, embeddability into hardware). & 5 months\\
    \hline
    \rowcolor{orange!50}
      Identify a set of benchmark ATM problems. & Hardness as a function of size. & 6.5 month\\
    \hline
    Analyze the solutions from the QUBO formulation with prior work. &
    Quality and diversity of solutions. & 8 month\\
    \hline
      Compile the ATM benchmark ensemble for the \DW chip at NASA Ames. & 
        Scaling of expected time to solution vs.\ size compared to classical code. Variety of different acceptable solutions. &10 month\\
    \hline
      Compare solutions from \DW chip with those from classical solvers & Potential quantum enhancement & 11 month\\
    \hline
    Assess scalability of ATM problems for different hardware (sizes, connectivity) and parameters (annealing strategy, etc.). &
        Potential quantum enhancement. & 12 month\\
    \hline
  \end{tabular}\caption{\label{table:milestone}Breakdown of the project effort into milestones, including suggested performance metric and completion
    dates; green and orange indicate completed and partially completed tasks, respectively.}
\end{table}



