% \section*{Project Aims and Approach}\label{sec:intro}
% The aim of this project is 
\noindent
{\bf Aim:} To provide an initial assessment of the potential 
of quantum annealing to attack challenging computational problems in the ATM 
domain. 
\linebreak
{\bf Approach:} This assessment will be based on a case study, involving runs
on the \DW\ quantum annealer and classical annealing simulations, in which 
a variety of quantum annealing approaches are developed and compared 
on benchmark problems derived from a specific real-world ATM problem.
\linebreak
{\bf Personnel:} This work is being performed by members of the QuAIL team (Tobias Stollenwerk, Bryan O'Gorman, Salvatore Mandr\`a, Davide Venturelli and Eleanor G. Rieffel) with expertise in applying quantum annealing to real-world problems, in close collaboration with ATM domain experts (Olga Rodionova, Hok K. Ng and Banavar Sridhar).
This work builds on the team's promising prior results in other planning and
scheduling domains~\cite{rieffel:15,venturelli:15}.
% See \tablename~\ref{table:milestone} for a complete overview of completed and future milestones.
\noindent
\begin{minipage}[t]{0.9\columnwidth}
\textbf{Projected final project outputs:}
\begin{itemize}[leftmargin=0.5cm]
  \itemsep-0.5em
  \item Programming toolkit to analyze trajectories and potential conflicts
  \item Mappings of the ATM problem to QUBO
  \item Implemented mappings on benchmark problems: taking a set of wind-optimal trajectories and outputting a QUBO instance
  \item Analysis of runs of benchmark instances on the \DW\ quantum annealer 
  \item Report summarizing initial assessment, with recommendations
for future research
\end{itemize}
\end{minipage}
